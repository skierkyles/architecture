\documentclass[12pt,a4paper]{report}
\usepackage[utf8]{inputenc}
\usepackage{amsmath}
\usepackage{amsfonts}
\usepackage{amssymb}
\usepackage{graphicx}
\usepackage{lmodern}

\linespread{2}

\author{Kyle Swanson}
\title{Lab 4: Logic Breadboard}

\begin{document}
\maketitle

\paragraph{}
For this lab, we focused heavily on breadboards. We had a simple launchpad application that would "walk" several IO ports from the binary number 0000 to 1111. The first step was representing this on the breadboard. We used LEDs in a vertical configuration as shown in figure 1. Starting from the right we had bit 0, then 1, and so on. \\
\begin{figure}
	\centering
	\includegraphics[scale=.1]{img/led_config} \\
	\caption{Our LED Arrangement}
\end{figure}

\paragraph{} 
The first step was building the circuits we needed for these experiments. We went through several iterations of our layout and colouring schemes when building. Quickly we learned how crucial the layout was in maintaining even a small breadboard. 

\end{document}