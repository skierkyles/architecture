\documentclass[12pt,a4paper]{report}
\usepackage[utf8]{inputenc}
\usepackage[fleqn]{amsmath}
\usepackage{amsfonts}
\usepackage{amssymb}
\author{Kyle Swanson}
\title{Chapter 1 Exercises: Number Systems}
\setlength{\parindent}{0cm}

\begin{document}

\maketitle

\begin{normalsize}



\textbf{1.8} What is the largest 32-bit unsigned number?\\
$ 2^{32}-1 = 4,294,967,295$ \\

\textbf{1.10} What is the largest 32-bit binary number that can be represented with\\
(a) unsigned numbers?\\
$ 2^{32}-1 = 4,294,967,295 $

(b) two's complement numbers?\\
$ 2^{31}-1 = 2,147,483,647 $

(c) sign/magnitude numbers?\\
$ 2^{31}-1 = 2,147,483,647 $ \\

\textbf{1.12} What is the smallest (most negative) 32-bit binary number that can be represented with\\
(a) unsigned numbers?\\
If you can't have a sign, either it's impossible to have a negative value, or you assume that all values are negative. A negative-int data type I suppose.
$ -2^{32}-1 = -4,294,967,295 $

(b) two's complement numbers?\\
$ -2^{31} = -2,147,483,648 $

(c) sign/magnitude numbers?\\
$ -2^{31}-1 = -2,147,483,647 $ \\

\textbf{1.14} Convert the following unsigned binary numbers to decimal.\\
(a) $ 1110_{2} $ \\
$ 2^{3} + 2^{2} + 2^{1} + 0 = 14 $ \\
(b) $ 10 0100_{2} $ \\
$ 2^{5} + 2^{2} = 36 $ \\
(c) $ 1101 0111_{2} $ \\
$ 2^{7} + 2^{6} + 2^{4} + 2^{2} + 2^{1} + 2{0} = 215 $ \\
(d) $ 011 1010 1010 0100_{2} $ \\
$ 2^{13} + 2^{12} + 2^{11} + 2^{9} + 2^{7} + 2^{5} + 2^{2} = 15,012 $ \\

\textbf{1.16} Repeat 1.14, but convert to hexadecimal. \\
Since they are split into 4 bit sections, just match each section with it's hex digit.\\
(a) $ 1110_{2} $ \\
$ 0xE $ \\
(b) $ 10\:0100_{2} $ \\
$ 0x24 $ \\
(c) $ 1101\:0111_{2} $ \\
$ 0xD7 $ \\
(d) $ 011\:1010\:1010\:0100_{2} $ \\
$ 0x3AA4 $ \\

\textbf{1.18} Convert the following hexadecimal numbers to decimal. \\
(a) $ 0x4E $ \\
$ 4*16^{1} + 14*16^{0} = 78 $ \\
(b) $ 0x7C $ \\
$ 7*16^{1} + 12*16{0} = 124 $ \\
(c) $ 0xED3A $ \\
$ 14*16^{3} + 13*16^{2} + 3*16^{1} + 10*16^{0} = 60,730 $ \\
(d) $ 0x403FB001 $ \\
$ 4*16^{7} + 3*16^{5} + 15*16^{4} + 11*16^{3} + 1*16^{0} = 1,077,915,649 $ \\

\textbf{1.20} Repeat 1.18, but convert to unsigned binary. \\
Simply take each digit, and match it to it's binary equivalent. Then push the resulting sections together. \\
(a) $ 0x4E $ \\
$ 0100\: 1110 $ \\
(b) $ 0x7C $ \\
$ 0111\: 1100 $ \\
(c) $ 0xED3A $ \\
$ 1110\: 1101\: 0011\: 1010 $ \\
(d) $ 0x403FB001 $ \\
$ 0100\: 0000\: 0011\: 1111\: 1011\: 0000\: 0000\: 0001 $ \\

\textbf{1.22} Convert the following two's complement binary numbers to decimal. \\
(a) $ 1110_{2} $ \\ 
The left most bit is 1, flip the bits, $ 1110 = 0001 $ then add 1. $ 0001 + 0001 = 0010 = 2 $ \\
Since the left most bit was 1, the result is negative. $ -2 $ \\
(b) $ 100011_{2} $ \\ 
$ 011100 + 000001 = 011101 $ \\
$ 2^{4} + 2^{3} + 2^{2} + 2^{0} = -29 $ \\
(c) $ 01001110_{2} $ \\ 
The left most bit is 0, so this is a positive number. Continue as normal. \\
$ 2^{6} + 2^{3} + 2^{2} + 2^{1} = 78 $ \\
(d) $ 10110101_{2} $ \\ 
$ 01001010 + 00000001 = 01001011 $ \\
$  2^{6} + 2^{3} + 2^{1} + 2^{0} = -75 $ \\

\textbf{1.26} Convert the following decimal numbers to unsigned binary numbers. \\
(a) 14 \\
$ 14/2 = 7 r 0 $ \\
$ 7/2 = 3 r 1 $ \\
$ 3/2 = 1 r 1 $ \\
$ 1/2 = 0 r 1 $ \\
$ 14 = 1110 $ \\

(b) 52 \\
$ 52/2 = 26 r 0 $ \\
$ 26/2 = 13 r 0 $ \\
$ 13/2 = 6 r 1 $ \\
$ 6/2 = 3 r 0 $ \\
$ 3/2 = 1 r 1 $ \\
$ 1/2 = 0 r 1 $ \\
$ 52 = 110100 $ \\

(c) 339 \\
$ 339/2 = 169 r 1 $ \\
$ 169/2 = 84 r 1 $\\
$ 84/2 = 42 r 0 $\\
$ 42/2 = 21 r 0 $\\
$ 21/2 = 10 r 1 $\\
$ 10/2 = 5 r 0 $\\
$ 5/2 = 2 r 1 $\\
$ 2/2 = 1 r 0 $\\
$ 1/2 = 0 r 1 $\\
$ 339 = 101010011 $\\

(d) 711 \\
$ 711/2 = 355 r 1 $ \\
$ 355/2 = 177 r 1 $ \\
$ 177/2 = 88 r 1 $ \\
$ 88/2 = 44 r 0 $ \\
$ 44/2 = 22 r 0 $ \\
$ 22/2 = 11 r 0 $ \\
$ 11/2 = 5 r 1 $ \\
$ 5/2 = 2 r 1 $ \\
$ 2/2 = 1 r 0 $ \\
$ 1/2 = 0 r 1 $ \\
$ 711 = 1011000111 $ \\ 

\textbf{1.28} Repeat Exercise 1.26, but convert to hexadecimal. \\
Since it's converted to binary already, you can start with that, then go to hex. \\
(a) 14 \\
$ 14 = 1110 = 0xE $ \\

(b) 52 \\
$ 52 = 0011\: 0100 = 0x34$ \\

(c) 339 \\
$ 339 = 0001\: 0101\: 0011 = 0x153 $\\

(d) 711 \\
$ 711 = 0010\: 1100\: 0111 = 0x2C7 $

\textbf{1.30} Convert the following decimal numbers to 8-bit two's complement numbers or indicate that the decimal number would overflow the range. \\


\textbf{1.34} Convert the following 4-bit two's complement numbers to 8-bit two's complement numbers. \\

\textbf{1.36} Repeat Exercise 1.34 if the numbers are unsigned rather than two's complement. \\








\end{normalsize}

\end{document}