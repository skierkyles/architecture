\documentclass[12pt,a4paper]{report}
\usepackage[utf8]{inputenc}
\usepackage{amsmath}
\usepackage{amsfonts}
\usepackage{amssymb}
\usepackage{graphicx}
\usepackage{lmodern}

\newcommand*{\al}{\overline{A}}
\newcommand*{\bl}{\overline{B}}
\newcommand*{\cl}{\overline{C}}
\newcommand*{\dl}{\overline{D}}


\setlength{\parindent}{0cm}

\author{Kyle Swanson}
\title{Chapter 2 Exercises: Combinational Logic and Boolean Algebra }

\begin{document}
\maketitle

\begin{normalsize}

\textbf{2.2} Write a Boolean equation in sum-of-products canonical form for each of the truth tables in Figure 2.81. \\

To determine this, look where Y = 1. Then just put complemented or not A, B, C, or D AND'ED together (Ex: If B = 0, complement it.). That is, turn the inputs to 1. And OR the results. \\

(a) $ \al{}B = 1 $ , $ A\overline{B} = 1 $, $ AB = 1 $. Therefore \\ 
$ Y = \overline{A}B + A\overline{B} + AB $ \\

(b) $ Y = \overline{A}\overline{B}C + \overline{A}B\overline{C} + \overline{A}BC + A\overline{B}\overline{C} + AB\overline{C}$ \\

(c) $ Y = \overline{A}\overline{B}C + AB\overline{C} + ABC $ \\

(d) $ Y = \overline{A}\overline{B}\overline{C}\overline{D} + \overline{A}\overline{B}C\overline{D} + \overline{A}\overline{B}CD + \overline{A}BC\overline{D} + \overline{A}BCD + A\overline{B}\overline{C}\overline{D} + A\overline{B}C\overline{D} $ \\

(e) $ Y = \overline{A}\overline{B}CD + \overline{A}BC\overline{D} + \overline{A}BCD + A\overline{B}\overline{C}\overline{D} + A\overline{B}\overline{C}D + A\overline{B}C\overline{D} + A\overline{B}CD $ \\

\textbf{2.4} Write a Boolean equation in product-of-sums canonical form for the truth tables in 2.81. \\

To determine this, look where Y = 0. Then just put complemented or not A, B, C, or D OR'ED together (Ex: If B = 0, do not complement it.). That is, turn the inputs to 0. And AND the results. \\

(a) $ Y = (A + B) $ \\

(b) $ Y = (A + B + C) (\overline{A} + B + \overline{C}) (\overline{A} + \overline{B} + \overline{C}) $ \\

(c) $ Y = (A + B + C) (A + \overline{B} + C) (A + \overline{B} + \overline{C}) (\overline{A} + B + C) (\overline{A} + B + \overline{C}) $ \\

(d) $ Y = (A + B + C + \overline{D}) (A + \overline{B} + C + D) (A + \overline{B} + C + \overline{D}) (\overline{A} + B + C + \overline{D}) (\overline{A} + B + \overline{C} + \overline{D}) (\overline{A} + \overline{B} + C + D) (\overline{A} + \overline{B} + C + \overline{D}) (\overline{A} + \overline{B} + \overline{C} + D) (\overline{A} + \overline{B} + \overline{C} + \overline{D}) $ \\

(e) $ Y = (A + B + C + D) (A + B + C + \overline{D}) (A + B + \overline{C} + D) (A + \overline{B} + C + D) (A + \overline{B} + C + \overline{D}) (\overline{A} + \overline{B} + C + D) (\overline{A} + \overline{B} + C + \overline{D}) (\overline{A} + \overline{B} + \overline{C} + D) (\overline{A} + \overline{B} + \overline{C} + \overline{D}) $ \\

\textbf{2.6} Minimize each of the Boolean equations from Exercise 2.2. \\
(a) $ Y = \overline{A}B + A\overline{B} + AB $ \\
$ = \al{}B + A(\bl{}+B) $ T5'\\
$ = \al{}B + A $ \\
Finally, you can remove the $ \al{} $. Due to T1 or T2. See the examples below for justification.\\
$ = A + B $ \\
Examples on why $ \al{} $ can be eliminated. \\
A = 1, B = 1. $ 0\: and\: 1\: or\: 1 = 1 \Leftrightarrow 1\: or\: 1 = 1 $ \\
A = 0, B = 1. $ 1\: and\: 1\: or\: 0 = 1 \Leftrightarrow 1\: or\: 0 = 1 $ \\
A = 1, B = 0. $ 0\: and\: 0\: or\: 1 = 1 \Leftrightarrow 0\: or\: 1 = 1 $ \\
A = 0, B = 0. $ 1\: and\: 0\: or\: 0 = 0 \Leftrightarrow 0\: or\: 0 = 0 $ \\
See how they're the same? Since $ \al{} $ and B are AND'ed together, $ \al{} $ doesn't matter since it has A on the other side of an OR. \\

(b) $ Y = \overline{A}\overline{B}C + \overline{A}B\overline{C} + \overline{A}BC + A\overline{B}\overline{C} + AB\overline{C}$ \\
$ = \al{}(\bl{}C + B\cl{} + BC) + \cl{}(A\bl{} + AB) $ T8 \\
$ = \al{}(C(\bl{}+B) + B\cl{}) + A\cl{} $ T5, T3 \\
$ = \al{}(B\cl{} + C) + A\cl{} $ \\
$ = \al{}(B + C) + A\cl{} $ \\
$ = \al{}B + \al{}C + A\cl{} $ \\

(c) $ Y = \overline{A}\overline{B}C + AB\overline{C} + ABC $ \\

(d) $ Y = \overline{A}\overline{B}\overline{C}\overline{D} + \overline{A}\overline{B}C\overline{D} + \overline{A}\overline{B}CD + \overline{A}BC\overline{D} + \overline{A}BCD + A\overline{B}\overline{C}\overline{D} + A\overline{B}C\overline{D} $ \\

(e) $ Y = \overline{A}\overline{B}CD + \overline{A}BC\overline{D} + \overline{A}BCD + A\overline{B}\overline{C}\overline{D} + A\overline{B}\overline{C}D + A\overline{B}C\overline{D} + A\overline{B}CD $ \\


\textbf{2.8} Sketch a reasonably simple combinational circuit implementing each of the functions from Exercise 2.6. \\

\textbf{2.14} Simplify the following Boolean equations using Boolean theorems. Check for correctness using a truth table or K-map. \\

(a) $ Y = \overline{A}BC + \overline{A}B\overline{C} $ \\


(b) $ Y = \overline{ABC} + A\overline{B} $ \\

(c) $ Y = ABC\overline{D} + A\overline{BCD} + (\overline{A + B + C + D}) $ \\

\textbf{2.16} Sketch a reasonably simple combinational circuit implementing each of the functions from Exercise 2.14.\\

\textbf{2.22} Prove that the following theorems are true using perfect induction. You need not prove their duals.\\

(a) The idempotency theorem (T3) \\

(b) The distributivity theorem (T8) \\

(c) The combining theorem (T10) \\

\end{normalsize}

\end{document}























