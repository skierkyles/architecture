\documentclass[12pt,a4paper]{report}
\usepackage[utf8]{inputenc}
\usepackage{amsmath}
\usepackage{amsfonts}
\usepackage{amssymb}
\usepackage{graphicx}
\usepackage{lmodern}
\author{Kyle Swanson}
\title{Chapter 6 Exercises: Architecture}

\setlength{\parindent}{0cm}

\begin{document}
\maketitle
\begin{normalsize}

\textbf{Exercise 6.4} - Repeat Exercise 6.3 for memory storage of a 32-bit word stored at memory word 15 in a byte-addressable memory. \\
a) What is the byte address of memory word 15? \\
b) What are the byte address that memory word 15 spans? \\
c) Draw the number 0xFF223344 stored at word 15 in both big-endian and little-endian machines. Your drawing should be similar to Figure 6.4. Clearly label the byte address corresponding to each data byte value. \\

\medskip

\textbf{Exercise 6.10} - Convert the following MIPS assembly code into machine language. Write the instructions in hexadecimal. \\
\textit{add \$t0, \$s0, \$s1} \\
This is an R-type instruction. Add \\
opcode = 000000 \\
rs = \$s0 = 16 = 10000 \\
rt = \$s1 = 17 = 10001 \\
rd = \$t0 = 8 = 01000 \\
shamt = 00000 \\
func = add = 100000 \\

Put it together: \\
000000 10000 10001 01000 00000 100000 \\
= \textbf{0x2114020} \\

\textit{lw \$t0, 0x20(\$t7)} \\
This is a I-type instruction. Load Word - lw rt, imm(rs) \\

opcode = 100011 \\
rs = \$t7 = 15 = 01111 \\
rt = \$t0 = 8 = 01000 \\
imm = 0x20 = 0000000000100000 \\

Put it together: \\
100011 01111 01000 0000000000100000 \\
= \textbf{0x8DE80020} \\

\textit{addi \$s0, \$0, -10} \\
This is a I-type instruction. Add Immediate - addi rt, rs, imm \\

opcode = 001000 \\
rs = 00000 \\
rt = 16 = 10000 \\
imm = -10 = 1111111111110110 \\

Put it together: \\
001000 00000 10000 1111111111110110 \\
= \textbf{0x2010FFF6} \\

\medskip

\textbf{Exercise 6.12} - Consider I-type instructions. \\
a) Which instructions from Exercise 6.10 are I-type instructions? \\
addi and lw are both I-type. \\

b) Sign-extend the 16-bit immediate of each instruction from part (a) so that it becomes a 32bit number. \\
lw immediate = 0000000000100000 \\
Sign extended = 0000000000000000 0000000000100000 = 0x20

addi immediate = 1111111111110110 \\
Sign extended = 1111111111111111 1111111111110110 = 0xFFFFFFF6



\medskip

\textbf{Exercise 6.14 - Do not complete the reverse engineering. Do not explain function. Just convert.} \\

0x20080000 = 001000 00000 01000 0000000000000000 \\
I-type \\
opcode = 001000 = addi \\
rs = 00000 = \$0 \\
rt = 01000 = \$t0 \\
imm = 0000000000000000 = 0 \\

addi \$t0, \$0, 0 \\

0x20090001 = 001000 00000 01001 0000000000000001 \\
I-type \\
opcode = 001000 = addi \\
rs = 00000 = \$0 \\
rt = 01001 = 9 = \$t1 \\
imm = 0000000000000001 = 1\\

addi \$t1, \$0, 1 \\ 

0x0089502A = 000000 00100 01001 01010 00000 101010 \\
R-type \\
opcode = 000000 \\
rs = 00100 = 4 = \$a0 \\
rt = 01001 = 9 = \$t1 \\
rd = 01010 = 10 = \$t2 \\
shamt = 00000 = 0 \\
func = 101010 = slt \\

slt \$t2, \$a0, \$t1 \\

0x15400003 = 000101 01010 00000 0000000000000011 \\
I-type \\
opcode = 000101 = bne \\
rs = 01010 = 10 = \$t2 \\
rt = 00000 = \$0 \\
imm = 0000000000000011 = 3 = 0x3 \\

bne \$t2, \$0, 0x3 \\

0x01094020 = 000000 01000 01001 01000 00000 100000 \\
R-type \\
opcode = 000000 \\
rs = 01000 = 8 = \$t0 \\
rt = 01001 = 9 = \$t1 \\
rd = 01000 = 8 = \$t0 \\
shamt = 00000 = 0 \\
func = 100000 = add \\

add \$t0, \$t0, \$t1

0x21290002 = 001000 01001 01001 0000000000000010 \\
I-type \\
opcode = 001000 = addi \\
rs = 01001 = 9 = \$t1 \\
rt = 01001 = 9 = \$t1 \\
imm = 0000000000000010 = 2\\

addi \$t1, \$t1, 2 \\

0x08100002 = 000010 00000100000000000000000010 \\
opcode = 000010 = j \\
label = 00000100000000000000000010 = 0x100002 \\

j 0x100002 \\

0x01001020 = 000000 01000 00000 00010 00000 100000 \\
R-type \\
opcode = 000000 \\
rs = 01000 = 8 = \$t0 \\
rt = 00000 = 0 = \$0 \\
rd = 00010 = 2 = \$v1  \\
shamt = 00000 = 0 \\
func = 100000 = add \\

add \$v1, \$t0, \$0 \\

0x03E00008 = 000000 11111 00000 00000 00000 001000 \\
R-type \\
R-type = opcode, rs, rt, rd, shamt, func
opcode = 000000 \\
rs = 11111 = 31 = \$ra \\
rt = 00000 = 0 = \$0 \\
rd = 00000 = 0 = \$0 \\
shamt = 00000 = 0 = \$0 \\
func = 001000 = jr \\

jr \$ra \\

\medskip

\textbf{Exercise 6.16} \\




\end{normalsize}
\end{document}